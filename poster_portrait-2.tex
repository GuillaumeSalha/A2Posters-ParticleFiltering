\documentclass[portrait,final,a0paper,fontscale=0.277]{baposter}

\usepackage{calc}
\usepackage{graphicx}
\usepackage{amsmath}
\usepackage{amssymb}
\usepackage{relsize}
\usepackage{multirow}
\usepackage{rotating}
\usepackage{bm}
\usepackage{url}

\usepackage{graphicx}
\usepackage{multicol}

%\usepackage{times}
%\usepackage{helvet}
%\usepackage{bookman}
\usepackage{palatino}
\usepackage[]{algorithm2e}
\usepackage{pseudocode}






\newcommand{\captionfont}{\footnotesize}

\graphicspath{{images/}{../images/}}
\usetikzlibrary{calc}

\newcommand{\SET}[1]  {\ensuremath{\mathcal{#1}}}
\newcommand{\MAT}[1]  {\ensuremath{\boldsymbol{#1}}}
\newcommand{\VEC}[1]  {\ensuremath{\boldsymbol{#1}}}
\newcommand{\Video}{\SET{V}}
\newcommand{\video}{\VEC{f}}
\newcommand{\track}{x}
\newcommand{\Track}{\SET T}
\newcommand{\LMs}{\SET L}
\newcommand{\lm}{l}
\newcommand{\PosE}{\SET P}
\newcommand{\posE}{\VEC p}
\newcommand{\negE}{\VEC n}
\newcommand{\NegE}{\SET N}
\newcommand{\Occluded}{\SET O}
\newcommand{\occluded}{o}




%%%%%%%%%%%%%%%%%%%%%%%%%%%%%%%%%%%%%%%%%%%%%%%%%%%%%%%%%%%%%%%%%%%%%%%%%%%%%%%%
%%%% Some math symbols used in the text
%%%%%%%%%%%%%%%%%%%%%%%%%%%%%%%%%%%%%%%%%%%%%%%%%%%%%%%%%%%%%%%%%%%%%%%%%%%%%%%%

%%%%%%%%%%%%%%%%%%%%%%%%%%%%%%%%%%%%%%%%%%%%%%%%%%%%%%%%%%%%%%%%%%%%%%%%%%%%%%%%
% Multicol Settings
%%%%%%%%%%%%%%%%%%%%%%%%%%%%%%%%%%%%%%%%%%%%%%%%%%%%%%%%%%%%%%%%%%%%%%%%%%%%%%%%
\setlength{\columnsep}{1.5em}
\setlength{\columnseprule}{0mm}

%%%%%%%%%%%%%%%%%%%%%%%%%%%%%%%%%%%%%%%%%%%%%%%%%%%%%%%%%%%%%%%%%%%%%%%%%%%%%%%%
% Save space in lists. Use this after the opening of the list
%%%%%%%%%%%%%%%%%%%%%%%%%%%%%%%%%%%%%%%%%%%%%%%%%%%%%%%%%%%%%%%%%%%%%%%%%%%%%%%%
\newcommand{\compresslist}{%
\setlength{\itemsep}{1pt}%
\setlength{\parskip}{0pt}%
\setlength{\parsep}{0pt}%
}

%%%%%%%%%%%%%%%%%%%%%%%%%%%%%%%%%%%%%%%%%%%%%%%%%%%%%%%%%%%%%%%%%%%%%%%%%%%%%%
%%% Begin of Document
%%%%%%%%%%%%%%%%%%%%%%%%%%%%%%%%%%%%%%%%%%%%%%%%%%%%%%%%%%%%%%%%%%%%%%%%%%%%%%
\renewcommand{\baselinestretch}{1.15}
\begin{document}

%%%%%%%%%%%%%%%%%%%%%%%%%%%%%%%%%%%%%%%%%%%%%%%%%%%%%%%%%%%%%%%%%%%%%%%%%%%%%%
%%% Here starts the poster
%%%---------------------------------------------------------------------------
%%% Format it to your taste with the options
%%%%%%%%%%%%%%%%%%%%%%%%%%%%%%%%%%%%%%%%%%%%%%%%%%%%%%%%%%%%%%%%%%%%%%%%%%%%%%
% Define some colors

%\definecolor{lightblue}{cmyk}{0.83,0.24,0,0.12}
\definecolor{lightblue}{rgb}{0.145,0.6666,1}

% Draw a video
\newlength{\FSZ}
\newcommand{\drawvideo}[3]{% [0 0.25 0.5 0.75 1 1.25 1.5]
   \noindent\pgfmathsetlength{\FSZ}{\linewidth/#2}
   \begin{tikzpicture}[outer sep=0pt,inner sep=0pt,x=\FSZ,y=\FSZ]
   \draw[color=lightblue!50!black] (0,0) node[outer sep=0pt,inner sep=0pt,text width=\linewidth,minimum height=0] (video) {\noindent#3};
   \path [fill=lightblue!50!black,line width=0pt] 
     (video.north west) rectangle ([yshift=\FSZ] video.north east) 
    \foreach \x in {1,2,...,#2} {
      {[rounded corners=0.6] ($(video.north west)+(-0.7,0.8)+(\x,0)$) rectangle +(0.4,-0.6)}
    }
;
   \path [fill=lightblue!50!black,line width=0pt] 
     ([yshift=-1\FSZ] video.south west) rectangle (video.south east) 
    \foreach \x in {1,2,...,#2} {
      {[rounded corners=0.6] ($(video.south west)+(-0.7,-0.2)+(\x,0)$) rectangle +(0.4,-0.6)}
    }
;
   \foreach \x in {1,...,#1} {
     \draw[color=lightblue!50!black] ([xshift=\x\linewidth/#1] video.north west) -- ([xshift=\x\linewidth/#1] video.south west);
   }
   \foreach \x in {0,#1} {
     \draw[color=lightblue!50!black] ([xshift=\x\linewidth/#1,yshift=1\FSZ] video.north west) -- ([xshift=\x\linewidth/#1,yshift=-1\FSZ] video.south west);
   }
   \end{tikzpicture}
}

\hyphenation{resolution occlusions}
%%
\begin{poster}%
  % Poster Options
  {
  % Show grid to help with alignment
  grid=false,
  % Column spacing
  colspacing=1em,
  % Color style
  bgColorOne=white,
  bgColorTwo=white,
  borderColor=lightblue,
  headerColorOne=black,
  headerColorTwo=lightblue,
  headerFontColor=white,
  boxColorOne=white,
  boxColorTwo=lightblue,
  % Format of textbox
  textborder=roundedleft,
  % Format of text header
  eyecatcher=true,
  headerborder=closed,
  headerheight=0.1\textheight,
%  textfont=\sc, An example of changing the text font
  headershape=roundedright,
  headershade=shadelr,
  headerfont=\Large\bf\textsc, %Sans Serif
  textfont={\setlength{\parindent}{1.5em}},
  boxshade=plain,
%  background=shade-tb,
  background=plain,
  linewidth=2pt
  }
  % Eye Catcher
  {\includegraphics[height=0.1em]{plot1.png}} 
  % Title
  {\bf\textsc{Sequential Monte Carlo Methods For Probabilistic Graphical Models}\vspace{0.5em}}
  % Authors
  {\textsc{Gautier Appert, Lionel Riou-Durand, Guillaume Salha}}
  % University logo
  {% The makebox allows the title to flow into the logo, this is a hack because of the L shaped logo.
    \includegraphics[height=6.5em]{logo_ENS.jpg}
  }

%%%%%%%%%%%%%%%%%%%%%%%%%%%%%%%%%%%%%%%%%%%%%%%%%%%%%%%%%%%%%%%%%%%%%%%%%%%%%%
%%% Now define the boxes that make up the poster
%%%---------------------------------------------------------------------------
%%% Each box has a name and can be placed absolutely or relatively.
%%% The only inconvenience is that you can only specify a relative position 
%%% towards an already declared box. So if you have a box attached to the 
%%% bottom, one to the top and a third one which should be in between, you 
%%% have to specify the top and bottom boxes before you specify the middle 
%%% box.
%%%%%%%%%%%%%%%%%%%%%%%%%%%%%%%%%%%%%%%%%%%%%%%%%%%%%%%%%%%%%%%%%%%%%%%%%%%%%%
    %
    % A coloured circle useful as a bullet with an adjustably strong filling
    \newcommand{\colouredcircle}{%
      \tikz{\useasboundingbox (-0.2em,-0.32em) rectangle(0.2em,0.32em); \draw[draw=black,fill=lightblue,line width=0.03em] (0,0) circle(0.18em);}}





%%%%%%%%%%%%%%%%%%%%%%%%%%%%%%%%%%%%%%%%%%%%%%%%%%%%%%%%%%%%%%%%%%%%%%%%%%%%%%
  \headerbox{5 - SMC Sampler}{name=problem,column=0,row=0,span = 2}{
%%%%%%%%%%%%%%%%%%%%%%%%%%%%%%%%%%%%%%%%%%%%%%%%%%%%%%%%%%%%%%%%%%%%%%%%%%%%%%

\begin{multicols}{2}
\noindent SMC algorithms are based on some sequential use of Importance Sampling.

\begin{itemize}
 \item  The aim is to deal with an untractable distribution in high dimension: $\pi_d(x_{1:d})=\frac{\gamma_d(x_{1:d})}{Z}$. 
 \item SMC sampler can be used for instance to sample from $\pi_d(x_{1:d})$ or to estimate the partition function $Z$.
\item We \textbf{sample iteratively from conditional proposals in lower dimension}, corresponding to cliques of the graph.
\item SMC methods include a \textbf{re-sampling} step to avoid exponential grow in the weights.
\end{itemize}
\columnbreak

\begin{algorithm}[H]
\label{alg:qquad}
\DontPrintSemicolon
\textit{Perform each step for $i=1,...,N$}\;
\Begin{
Sample $X_{\mathcal{L}_1}^i \sim r_1(.)$\;
Set $w_1^i = \gamma_1(X_{\mathcal{L}_1}^i)/r_1(X_{\mathcal{L}_1}^i)$\;
\For{$k = 2,...,K$}{
Sample $a_k^i$ according to $\mathbb{P}(a_k^i =j) = \frac{\nu_{k-1}^j w_{k-1}^j}{\sum_l \nu_{k-1}^l w_{k-1}^l}$\;
Sample $\xi_k^i \sim r_k(.|X_{\mathcal{L}_{k-1}}^{a_k^i})$\;
Set $X_{\mathcal{L}_k}^i = X_{\mathcal{L}_{k-1}}^{a_k^i} \cup \xi_k^i$\;
Set $w_k^i = W_k(X_{\mathcal{L}_k}^i)$
}
}
\caption{Sequential Monte Carlo}
\end{algorithm}


\end{multicols}
   \vspace{0.3em}
   
   


 }







%%%%%%%%%%%%%%%%%%%%%%%%%%%%%%%%%%%%%%%%%%%%%%%%%%%%%%%%%%%%%%%%%%%%%%%%%%%%%%
  \headerbox{7 - Application to Classical XY model}{name=contribution,column=0,below=problem, span=2}{
%%%%%%%%%%%%%%%%%%%%%%%%%%%%%%%%%%%%%%%%%%%%%%%%%%%%%%%%%%%%%%%%%%%%%%%%%%%%%%
\begin{multicols}{2}

\begin{center}
\includegraphics[width=0.55\linewidth]{xy.png}
\end{center}

\begin{itemize}
    \item \textbf{XY Model}: A 2-dimensional d-square lattice model, extension of Ising model
    \item Set of cliques: pairs of adjacent vertices $x_i$ and $x_j$
    \item Joint probability distribution:
\end{itemize}
%The classical XY model is  $\mathcal{E}$. 
%Given the particular structure of the XY model, t which reduces the joint probability distribution to be of the simple form :
$$
p(X_{\mathcal{V}})  \propto \exp\left\{ \beta \displaystyle\sum_{i \sim j} \cos(x_i-x_j)   \right\}
$$
%To approximate iteratively our sequence of distributions $(\gamma_t)_t$, we add one variable at a time and all the associated neighbours. 
\textbf{Several decompositions} are used to sequence the graph :  left-right decomposition, random decomposition...
\begin{center}
\includegraphics[width=0.8\linewidth]{plot4.png}
\end{center}


\noindent We observe in practice that \textbf{the ordering has an important impact on the performance} of the SMC sampler.

\columnbreak

% je m'occupe du plot ! plot OK !!!


%Sequence of intermediate target distribution $(\gamma_t)_t$ is given :
%$$
%\gamma_k(X_{\mathcal{L}_k}) \propto 
%\gamma_{k-1}(X_{\mathcal{L}_{k-1}}) \hspace{0.1cm} 
%\exp\left\{ \kappa_k(X_{\mathcal{L}_{k-1}}) \cos(x_k-\mu_k(X_{\mathcal{L}_{k-1}}))   \right\}
%$$
%This leads us to choose the so-called 
% distribution to be our proposal distribution is \textit{von Mises} whose probability density function is given by :
%$$
%q_k(x_k|X_{\mathcal{L}_{k-1}}) = \frac{\exp\left\{ \kappa_k(X_{\mathcal{L}_{k-1}}) \cos(x_k-\mu_k(X_{\mathcal{L}_{k-1}}))   \right\}
%}{2\pi I_{0}( \kappa_k(X_{\mathcal{L}_{k-1}}))}
%$$

%\begin{figure}[h]
\begin{center}
\includegraphics[width=0.66\linewidth]{MSE.pdf}

Orderings and M.S.E. evolution
\end{center}
\begin{center}
\includegraphics[width=0.72\linewidth]{ESS.pdf}

Evolution of E.S.S. measure 

(re-sampling under 50\%)
\end{center}
\end{multicols}
   \vspace{0.3em}
  }











\headerbox{6 - Partition function}{name=smc,column=2,span=1,row=0}{
  %%%%%%%%%%%%%%%%%%%%%%%%%%%%%%%%%%%%%%%%%%%%%%%%%%%%%%%%%%%%%%%%%%%%%%%%%%%%%%

\noindent The \textbf{partition function} - normalizing constant - is a very interesting quantity in many applications (see $[1]$), such as:\begin{itemize}
    \item likelihood-based learning of parameters of the PGM
    \item derivation of free energy of a system of objects in mechanics
\end{itemize}\newline

\noindent Our Sequential Monte Carlo approach provides an \textbf{unbiased estimator} of this partition function, given by:
$$ \hat{Z}_{k}^{N} = \left( \frac{1}{N}  \displaystyle\sum_{i=1}^{N} \omega_{k}^{i} \right)
\left\{ \displaystyle\prod_{l=1}^{k-1} \frac{1}{N} \displaystyle\sum_{i=1}^{N} \nu_{l}^{i}  \omega_{l}^{i} \right\}$$
\newline 
\noindent This result is not straightforward. Therefore, the proof is reported in our final report.

      \vspace{-0.1em}
}





  \headerbox{8 - To go further}{name=background model,column=2, span = 1, below=smc}{
%%%%%%%%%%%%%%%%%%%%%%%%%%%%%%%%%%%%%%%%%%%%%%%%%%%%%%%%%%%%%%%%%%%%%%%%%%%%%%
\noindent We presented a new framework for inference in PGM using Sequential Monte Carlo. \newline

\noindent Although the NIPS 2014 paper is not the first tentative on this topic, previous work were:\begin{itemize}
    \item only designed for Gaussian or discrete interactions between variables [Hamze and De Freitas (2005); Everitt (2012)]
    \item and-or did not develop such sequential decomposition of the graph [Isard, 2003; Frank \textit{et al.},2009]
\end{itemize}  \newline

\noindent Besides XY model, the authors also introduced two others applications for their SMC sampler:\begin{itemize}
    \item a simple toy \textbf{Gaussian Markov Random Field} model to incorporate \textbf{Particle MCMC} sampling
    \item a \textbf{Latent Dirichlet Allocation} - directed - PGM, to perform likelihood estimation in topic models
\end{itemize} \newline

\noindent We provide a quick review of these applications in the final report.
   \vspace{0.3em}
  }
%%%%%%%%%%%%%%%%%%%%%%%%%%%%%%%%%%%%%%%%%%%%%%%%%%%%%%%%%%%%%%%%%%%%%%%%%%%%%%




%%%%%%%%%%%%%%%%%%%%%%%%%%%%%%%%%%%%%%%%%%%%%%%%%%%%%%%%%%%%%%%%%%%%%%%%%%%%%%
  \headerbox{References}{name=references,column=0,span=2, above=bottom}{
%%%%%%%%%%%%%%%%%%%%%%%%%%%%%%%%%%%%%%%%%%%%%%%%%%%%%%%%%%%%%%%%%%%%%%%%%%%%%%
    \smaller
    \bibliographystyle{ieee}
    \renewcommand{\section}[2]{\vskip 0.05em}
      \begin{thebibliography}{1}\itemsep=-0.01em
      \setlength{\baselineskip}{0.4em}
      \bibitem{nips:nipssmc}
      Naesseth, C. A., Lindsten, F., \& Schon, T. B. (2014). Sequential Monte Carlo for graphical models. \textit{In Advances in Neural Information Processing Systems}, pp. 1862-1870.
        \bibitem{doucet:tuto}
        Doucet, A., \& Johansen, A. M. (2009). A tutorial on particle filtering and smoothing: Fifteen years later. \textit{Handbook of Nonlinear Filtering}, 12, 656-704.
        \bibitem{robert:mcmc}
        Robert, C., & Casella, G. (2004). Monte Carlo statistical methods. \textit{Springer Science & Business Media}.
        \bibitem{xy:tuto}
        Chaikin, P. M., & Lubensky, T. C. (2000). Principles of condensed matter physics. \textit{Cambridge: Cambridge university press}.
      \end{thebibliography}
   \vspace{0.3em}
  }
%%%%%%%%%%%%%%%%%%%%%%%%%%%%%%%%%%%%%%%%%%%%%%%%%%%%%%%%%%%%%%%%%%%%%%%%%%%%%%



%%%%%%%%%%%%%%%%%%%%%%%%%%%%%%%%%%%%%%%%%%%%%%%%%%%%%%%%%%%%%%%%%%%%%%%%%%%%%%
  \headerbox{Source Code}{name=source,column=2,span = 1, above=bottom}{
%%%%%%%%%%%%%%%%%%%%%%%%%%%%%%%%%%%%%%%%%%%%%%%%%%%%%%%%%%%%%%%%%%%%%%%%%%%%%%
  \noindent
  \begin{minipage}{\linewidth}
  \begin{minipage}{0.5\linewidth}
    \indent{}All implementations have been made with R. The entire source code of this project will be available on the following GitHub link: \\
  \end{minipage}\hfill%
  \begin{minipage}{0.48\linewidth}
  \hfill\includegraphics[width=\linewidth]{flashcode.png}
  \end{minipage}
  \end{minipage}
  \url{https://github.com/GuillaumeSalha/ParticleFiltering}
  }
%%%%%%%%%%%%%%%%%%%%%%%%%%%%%%%%%%%%%%%%%%%%%%%%%%%%%%%%%%%%%%%%%%%%%%%%%%%%%%

\end{poster}

\end{document}