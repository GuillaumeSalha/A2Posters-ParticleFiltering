\documentclass[portrait,final,a0paper,fontscale=0.277]{baposter}

\usepackage{calc}
\usepackage{graphicx}
\usepackage{amsmath}
\usepackage{amssymb}
\usepackage{relsize}
\usepackage{multirow}
\usepackage{rotating}
\usepackage{bm}
\usepackage{url}

\usepackage{graphicx}
\usepackage{multicol}

%\usepackage{times}
%\usepackage{helvet}
%\usepackage{bookman}
\usepackage{palatino}

\newcommand{\captionfont}{\footnotesize}

\graphicspath{{images/}{../images/}}
\usetikzlibrary{calc}

\newcommand{\SET}[1]  {\ensuremath{\mathcal{#1}}}
\newcommand{\MAT}[1]  {\ensuremath{\boldsymbol{#1}}}
\newcommand{\VEC}[1]  {\ensuremath{\boldsymbol{#1}}}
\newcommand{\Video}{\SET{V}}
\newcommand{\video}{\VEC{f}}
\newcommand{\track}{x}
\newcommand{\Track}{\SET T}
\newcommand{\LMs}{\SET L}
\newcommand{\lm}{l}
\newcommand{\PosE}{\SET P}
\newcommand{\posE}{\VEC p}
\newcommand{\negE}{\VEC n}
\newcommand{\NegE}{\SET N}
\newcommand{\Occluded}{\SET O}
\newcommand{\occluded}{o}

%%%%%%%%%%%%%%%%%%%%%%%%%%%%%%%%%%%%%%%%%%%%%%%%%%%%%%%%%%%%%%%%%%%%%%%%%%%%%%%%
%%%% Some math symbols used in the text
%%%%%%%%%%%%%%%%%%%%%%%%%%%%%%%%%%%%%%%%%%%%%%%%%%%%%%%%%%%%%%%%%%%%%%%%%%%%%%%%

%%%%%%%%%%%%%%%%%%%%%%%%%%%%%%%%%%%%%%%%%%%%%%%%%%%%%%%%%%%%%%%%%%%%%%%%%%%%%%%%
% Multicol Settings
%%%%%%%%%%%%%%%%%%%%%%%%%%%%%%%%%%%%%%%%%%%%%%%%%%%%%%%%%%%%%%%%%%%%%%%%%%%%%%%%
\setlength{\columnsep}{1.5em}
\setlength{\columnseprule}{0mm}

%%%%%%%%%%%%%%%%%%%%%%%%%%%%%%%%%%%%%%%%%%%%%%%%%%%%%%%%%%%%%%%%%%%%%%%%%%%%%%%%
% Save space in lists. Use this after the opening of the list
%%%%%%%%%%%%%%%%%%%%%%%%%%%%%%%%%%%%%%%%%%%%%%%%%%%%%%%%%%%%%%%%%%%%%%%%%%%%%%%%
\newcommand{\compresslist}{%
\setlength{\itemsep}{1pt}%
\setlength{\parskip}{0pt}%
\setlength{\parsep}{0pt}%
}

%%%%%%%%%%%%%%%%%%%%%%%%%%%%%%%%%%%%%%%%%%%%%%%%%%%%%%%%%%%%%%%%%%%%%%%%%%%%%%
%%% Begin of Document
%%%%%%%%%%%%%%%%%%%%%%%%%%%%%%%%%%%%%%%%%%%%%%%%%%%%%%%%%%%%%%%%%%%%%%%%%%%%%%
\renewcommand{\baselinestretch}{1.15}
\begin{document}

%%%%%%%%%%%%%%%%%%%%%%%%%%%%%%%%%%%%%%%%%%%%%%%%%%%%%%%%%%%%%%%%%%%%%%%%%%%%%%
%%% Here starts the poster
%%%---------------------------------------------------------------------------
%%% Format it to your taste with the options
%%%%%%%%%%%%%%%%%%%%%%%%%%%%%%%%%%%%%%%%%%%%%%%%%%%%%%%%%%%%%%%%%%%%%%%%%%%%%%
% Define some colors

%\definecolor{lightblue}{cmyk}{0.83,0.24,0,0.12}
\definecolor{lightblue}{rgb}{0.145,0.6666,1}

% Draw a video
\newlength{\FSZ}
\newcommand{\drawvideo}[3]{% [0 0.25 0.5 0.75 1 1.25 1.5]
   \noindent\pgfmathsetlength{\FSZ}{\linewidth/#2}
   \begin{tikzpicture}[outer sep=0pt,inner sep=0pt,x=\FSZ,y=\FSZ]
   \draw[color=lightblue!50!black] (0,0) node[outer sep=0pt,inner sep=0pt,text width=\linewidth,minimum height=0] (video) {\noindent#3};
   \path [fill=lightblue!50!black,line width=0pt] 
     (video.north west) rectangle ([yshift=\FSZ] video.north east) 
    \foreach \x in {1,2,...,#2} {
      {[rounded corners=0.6] ($(video.north west)+(-0.7,0.8)+(\x,0)$) rectangle +(0.4,-0.6)}
    }
;
   \path [fill=lightblue!50!black,line width=0pt] 
     ([yshift=-1\FSZ] video.south west) rectangle (video.south east) 
    \foreach \x in {1,2,...,#2} {
      {[rounded corners=0.6] ($(video.south west)+(-0.7,-0.2)+(\x,0)$) rectangle +(0.4,-0.6)}
    }
;
   \foreach \x in {1,...,#1} {
     \draw[color=lightblue!50!black] ([xshift=\x\linewidth/#1] video.north west) -- ([xshift=\x\linewidth/#1] video.south west);
   }
   \foreach \x in {0,#1} {
     \draw[color=lightblue!50!black] ([xshift=\x\linewidth/#1,yshift=1\FSZ] video.north west) -- ([xshift=\x\linewidth/#1,yshift=-1\FSZ] video.south west);
   }
   \end{tikzpicture}
}

\hyphenation{resolution occlusions}
%%
\begin{poster}%
  % Poster Options
  {
  % Show grid to help with alignment
  grid=false,
  % Column spacing
  colspacing=1em,
  % Color style
  bgColorOne=white,
  bgColorTwo=white,
  borderColor=lightblue,
  headerColorOne=black,
  headerColorTwo=lightblue,
  headerFontColor=white,
  boxColorOne=white,
  boxColorTwo=lightblue,
  % Format of textbox
  textborder=roundedleft,
  % Format of text header
  eyecatcher=true,
  headerborder=closed,
  headerheight=0.1\textheight,
%  textfont=\sc, An example of changing the text font
  headershape=roundedright,
  headershade=shadelr,
  headerfont=\Large\bf\textsc, %Sans Serif
  textfont={\setlength{\parindent}{1.5em}},
  boxshade=plain,
%  background=shade-tb,
  background=plain,
  linewidth=2pt
  }
  % Eye Catcher
  {\includegraphics[height=0.1em]{plot1.png}} 
  % Title
  {\bf\textsc{Sequential Monte Carlo Methods For Probabilistic Graphical Models}\vspace{0.5em}}
  % Authors
  {\textsc{Gautier Appert, Lionel Riou-Durand, Guillaume Salha}}
  % University logo
  {% The makebox allows the title to flow into the logo, this is a hack because of the L shaped logo.
    \includegraphics[height=6.5em]{logo_ENS.jpg}
  }

%%%%%%%%%%%%%%%%%%%%%%%%%%%%%%%%%%%%%%%%%%%%%%%%%%%%%%%%%%%%%%%%%%%%%%%%%%%%%%
%%% Now define the boxes that make up the poster
%%%---------------------------------------------------------------------------
%%% Each box has a name and can be placed absolutely or relatively.
%%% The only inconvenience is that you can only specify a relative position 
%%% towards an already declared box. So if you have a box attached to the 
%%% bottom, one to the top and a third one which should be in between, you 
%%% have to specify the top and bottom boxes before you specify the middle 
%%% box.
%%%%%%%%%%%%%%%%%%%%%%%%%%%%%%%%%%%%%%%%%%%%%%%%%%%%%%%%%%%%%%%%%%%%%%%%%%%%%%
    %
    % A coloured circle useful as a bullet with an adjustably strong filling
    \newcommand{\colouredcircle}{%
      \tikz{\useasboundingbox (-0.2em,-0.32em) rectangle(0.2em,0.32em); \draw[draw=black,fill=lightblue,line width=0.03em] (0,0) circle(0.18em);}}

%%%%%%%%%%%%%%%%%%%%%%%%%%%%%%%%%%%%%%%%%%%%%%%%%%%%%%%%%%%%%%%%%%%%%%%%%%%%%%
  \headerbox{1 - Objectives}{name=problem,column=0,row=0}{
%%%%%%%%%%%%%%%%%%%%%%%%%%%%%%%%%%%%%%%%%%%%%%%%%%%%%%%%%%%%%%%%%%%%%%%%%%%%%%
\noindent General context:
\begin{itemize}
\item Inference in statistical models involving a large number of latent variables is in general a difficult problem
\item{Probabilistic Graphical Models} are a relevant and useful way to tackle this issue in many cases
\end{itemize}

\noindent In this project, we focus on \textbf{Sequential Monte Carlo methods} to solve \textbf{inference problems} in graphical models:
\begin{itemize}
\item We provide a review of the NIPS 2014 article of Naesseth, Lindsten and Sch{\"o}m $[1]$
\item We explain their new framework to make use of SMC methods in PGM
\item We implement the method on Classical XY Model
\end{itemize}
   \vspace{0.3em}
 }

%%%%%%%%%%%%%%%%%%%%%%%%%%%%%%%%%%%%%%%%%%%%%%%%%%%%%%%%%%%%%%%%%%%%%%%%%%%%%%
  \headerbox{2 - Graphical Models}{name=contribution,column=0,below=problem, above=bottom}{
%%%%%%%%%%%%%%%%%%%%%%%%%%%%%%%%%%%%%%%%%%%%%%%%%%%%%%%%%%%%%%%%%%%%%%%%%%%%%%
    \noindent We consider \textbf{undirected models} of the form:
    $$p(X_{\mathcal{V}}) = \frac{1}{Z} \displaystyle\prod_{c \in \mathcal{C}} \psi_{c}(X_c),$$
    where the graph  $\mathcal{G} = \{\mathcal{V}, \mathcal{E}\}$ has vertex set $\mathcal{V} = \{X_1,...,X_{|\mathcal{V}|}\}$, edge set $\mathcal{E}$, cliques $\mathcal{C}$ and $Z = \sum_X \displaystyle\prod_{c \in \mathcal{C}} \psi_{c}(X_c)$ is the \textbf{partition function} (normalizing constant).
    \newline
    
    \noindent To make interactions between variables more explicit, $[1]$ define a \textbf{factor graph} $\mathcal{F} = \{\mathcal{V},\Psi,\mathcal{E}'\}$, associated to $\mathcal{G}$:
    \begin{itemize}
    \item Vertices: original r.v. + factors $\Psi = \{\psi_c : c \in \mathcal{C}\}$
    \item Edges: from variables to factors
    \end{itemize}

    
\noindent For instance, the following graphical model:
    
    \includegraphics[width=0.75\linewidth]{plot1.png}
    
   \vspace{0.3em}
   
   \noindent could be associated to the following factor graph: 
   
   \includegraphics[width=0.9\linewidth]{plot2.png}
  }

%%%%%%%%%%%%%%%%%%%%%%%%%%%%%%%%%%%%%%%%%%%%%%%%%%%%%%%%%%%%%%%%%%%%%%%%%%%%%%
\headerbox{3 - Sequential Decomposition of Graphical Models}{name=results,column=1,span=2,row=0}{
  %%%%%%%%%%%%%%%%%%%%%%%%%%%%%%%%%%%%%%%%%%%%%%%%%%%%%%%%%%%%%%%%%%%%%%%%%%%%%%
  \begin{multicols}{2}
  \noindent In order to use SMC methods for inference in graphical models, it is necessary to define a \textbf{sequence of target probability distributions} to approximate:
  \begin{itemize}
  \item We could intuitively want to use marginal distributions under $p(X_{\mathcal{V}})$. However, this is not an obligation ($[1]$)
  \item The choice of target distribution is theoretically quite arbitrary as long as, at some final iteration, we recover
$p(X_{\mathcal{V}})$
\item In $[1]$, they obtain such a sequence by doing a \textbf{sequential decomposition} of the graph
\end{itemize}

\columnbreak

\noindent  \textbf{Sequential decomposition:}
Define a sequence of factors $\{\psi_k\}^K_{k=1}$ such that
$$\psi_k(X_{\mathcal{I}_k}) = \displaystyle\prod_{c \in \mathcal{C}_k} \psi_{c}(X_c),$$
where the different $\mathcal{C}_k \subset \mathcal{C}$ verify
$\bigcup_{k=1}^K \mathcal{C}_k = \mathcal{C}, \hspace{0.3cm}$
and where $\mathcal{I}_k = \bigcup_{c \in \mathcal{C}_k} c \subseteq \{1,...,|\mathcal{V}|\}.$

\noindent Therefore, the factorizarion of $p$ can be rewritten as:
$$p(X_{\mathcal{V}}) = \frac{1}{Z} \displaystyle\prod_{k = 1}^K \psi_k(X_{\mathcal{I}_k}).$$

We report a possible decomposition of the previous factor graph:

\end{multicols}


\begin{center}
  \includegraphics[width=0.75\linewidth]{plot3.png}
  \end{center}
      \begin{multicols}{2}
\noindent $[1]$ denote the auxiliary quantities
by $\tilde{\gamma}_k(X_{\mathcal{L}_k}) := \prod_{l = 1}^k \psi_l(X_{\mathcal{I}_l})$, with $\mathcal{L}_k := \cup_{l=1}^k \mathcal{I}_l$ and 
$\mathcal{L}_K = \mathcal{V}$. So:
$$p(X_{\mathcal{L}_K}) \propto \tilde{\gamma}_K(X_{\mathcal{L}_K}).$$
Therefore, we obtain the correct target distribution at time $K$.
\columnbreak

\noindent The corresponding normalized probability distribution functions are given by 
$\bar{\gamma}_k(X_{\mathcal{L}_k}) = \tilde{\gamma}_k(X_{\mathcal{L}_k})/Z_k$ with $Z_k = \int \tilde{\gamma}_k(X_{\mathcal{L}_k}) d X_{\mathcal{L}_k}$. 

\noindent An important requirement:
$$\int \tilde{\gamma}_k(X_{\mathcal{L}_k}) d X_{\mathcal{L}_k} < +\infty, 
\forall\ k \in \{1,...,K\}$$

\end{multicols}
      \vspace{-0.6em}
}






%%%%%%%%%%%%%%%%%%%%%%%%%%%%%%%%%%%%%%%%%%%%%%%%%%%%%%%%%%%%%%%%%%%%%%%%%%%%%%
  \headerbox{4 - Importance Sampling}{name=background model,column=1,, span = 2, above=bottom}{
%%%%%%%%%%%%%%%%%%%%%%%%%%%%%%%%%%%%%%%%%%%%%%%%%%%%%%%%%%%%%%%%%%%%%%%%%%%%%%
\begin{multicols}{2}

  \indent Sequential Monte-Carlo methods relate to \textbf{Importance Sampling}, a well known Monte-Carlo method for:
  \begin{itemize}
      \item Approximating intractable integrals
      \item More generally, sample indirectly from a target distribution
  \end{itemize}
  
  Let $\pi$ be an intractable density, $(X^{(1)},\dots,X^{(N)})$ be i.i.d. samples from a positive \textbf{proposal distribution} $q$ from which we can sample. We define the importance weights as $w(x) := \frac{\gamma(x)}{q(x)}$. \columnbreak
  
  \noindent The \textbf{normalized weights} are $W(X^{(i)}):=\frac{w(X^{(i)})}{\sum_{j=1}^Nw(X^{(j)})}$.
We derive using the LLN:
$$
\hat{\mathcal{I}}_{\text{NIS}} := \sum_{i=1}^{N} h(X^{(i)}) W(X^{(i)})
\overset{\text{p.s}}{\longrightarrow} \mathbb{E}_{X \sim \pi}[h(X)]
$$
\noindent It also provides a \textbf{point-mass approximation} of the target distribution.

\noindent However, IS collapses when the dimension is too large as the variance of the weights grows exponentially. To tackle this issue, we describe the SMC sampler, in section 5.
  \end{multicols}
   \vspace{0.3em}
  }
%%%%%%%%%%%%%%%%%%%%%%%%%%%%%%%%%%%%%%%%%%%%%%%%%%%%%%%%%%%%%%%%%%%%%%%%%%%%%%
%%%%%%%%%%%%%%%%%%%%%%%%%%%%%%%%%%%%%%%%%%%%%%%%%%%%%%%%%%%%%%%%%%%%%%%%%%%%%%

\end{poster}

\end{document}

